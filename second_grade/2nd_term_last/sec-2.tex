\documentclass[dvipdfmx]{jsarticle}
\usepackage[dvipdfmx]{graphicx}
\begin{document}
\title{期末テスト復習}
\begin{enumerate}
  \item 問の角度を書きなさい。 
    \begin{enumerate}
      \item xとy 
        \vspace{0.2in} \\*
        \includegraphics[width=7cm]{g4750.png}
        \vspace{0.2in} \\*
        \underline{x =   \hspace{0.8in} y =   }
        \vspace{0.2in}
      \item x \\*
        \includegraphics[width=5cm]{s12.png}
        \vspace{0.2in} \\*
        \underline{x = }
        \vspace{0.2in} \\*
        \newpage
      \item x
        \vspace{0.2in} \\*
        \includegraphics[width=5cm]{g3877.png}
        \vspace{0.2in} \\*
        \underline{x = }

      \item x
        \vspace{0.2in} \\*
        \includegraphics[width=5cm]{g5090.png}
        \vspace{0.2in} \\*
        \underline{x = }
      \item x,y
        \vspace{0.2in} \\*
        \includegraphics[width=5cm]{g5091.png}
        \vspace{0.2in} \\*
        \underline{x = \hspace{0.8in } y = }
    \end{enumerate}
    \newpage
  \item 証明しなさい
    \begin{enumerate}
      \item 星型五角形の5つの核の和は180°である。
        \vspace{0.2in} \\*
        \includegraphics[width=5cm]{path4676.png}
        \vspace{2in} \\*
      \item AB=AC, AD=AE であるとき、角B と角cは等しい
        \vspace{0.2in} \\*
        \includegraphics{g3857.png}
        \vspace{2in}

      \item AD // BC 角ABCの二等分線の延長をDとすると、AB = ADとなることを証明
        \vspace{0.2in}
        \includegraphics{g4196.png}

        \vspace{2in}
      \item Cが角XOYの二等分線であることを証明

      \includegraphics{g5088.png}
    \end{enumerate}
  \end{enumerate}
\end{document}
