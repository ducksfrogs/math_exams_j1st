\documentclass[twocolumn, fleqn]{jsarticle}
\setlength{\columnseprule}{0.4pt}
\renewcommand{\labelenumi}{\textbf{\theenumi .}}
\renewcommand{\theenumii}{\arabic{enumii}}

\setlength{\mathindent}{0zw}
\pagestyle{empty}

\begin{document}
\twocolumn[
  \begin{center}
    \LARGE\textbf{2回目}
  \end{center}
  \begin{flushright}
    1st\underline{\hspace{3zw}}class\underline{\hspace{3zw}}\hspace{1zw}name \underline{\hspace{13zw}}
  \end{flushright}
  ]



  \begin{enumerate}
    \item 次の式を、文字式の表し方に従って書きなさい 

      \begin{enumerate}
        \item  $ b \times c $
          \vfill
        \item  $x \times y \times a$
          \vfill
        \item  $ (x + y) \times 5 $
          \vfill
        \item $(a-b) \times 8$
          \vfill
        \item $ a \times b \times b \times 4$
          \vfill
        \item $ (a +4)\div  3$
          \vfill
        \item $ (2a +8)\div  2$
          \vfill
        \item $ 3x + 4x$
          \vfill
      \end{enumerate}

    \newpage
  \item 次の計算をしなさい 

      \begin{enumerate}
        \item  $6x -7 +3x -2$
          \vfill
        \item  $ a + 5 -7a -5$
          \vfill
        \item  $x^2 +5 +x + 2x -7$
          \vfill
        \item $\frac{3}{5}x \div(- \frac{2}{15})$
          \vfill
        \item $2(x+3)$
          \vfill
      \end{enumerate}
    \item $x$に3を$y$に2を代入して計算をしなさい 
      \begin{enumerate}
          \item $ 3x +2y $
            \vfill
          \item $ x^2 +2 $
            \vfill
      \end{enumerate}

  \end{enumerate}


\end{document}



